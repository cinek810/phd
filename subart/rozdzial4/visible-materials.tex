Omawiane w niniejszym rozdziale zjawiska wykorzystują zakresy długości fali w których materiały charakteryzują się nietypowymi własnościami. Istenienie takich zakresów spektralnych wynika ze zjawiska dyspersji czasowej materiałów. Zjawisko dyspersji jest przejawem występowania ładunków elektrycznych w budowie materii. Cząstki posiadające ładunek elektryczny jak elektrony lub jony, wprowadzane są w drgania pod wpływem fali elektromagnetycznej. Przez co stają się one źródłem promieniowania elektromagnetycznego modyfikującego fale propagującą się w ośrodku. Ze względu na występowanie częstości własnych drgań cząstek ośrodka zjawiska dyspersyjne mają charakter rezonansowy.

Poniższy rozdział poświęcony jest analizie zjawisk w zakresie optycznym w któym podatność magenetyczna materiałów jest zaniedbywalnie mała, w związku z czym przyjmuje się $\mu(\omega)=\textrm{const}$. Przenikalność elektryczna materiałów wykazuje jednak zmienność również dla widma optycznego. W stosunku do dielktryków z powodzeniem stosuje się model Lorenza-Lorenza, a wartość $\varepsilon$ w prezentowanych zastosowaniach może być traktowana jako stała. Bardziej skompilikowane podejście jest jednak wymagane w przypadku metali, dla których $\varepsilon(\omega)$ w omawianej części spektrum ma charakter rezonansowy i musi być opisywana przy pomocy modelu Lorenza-Drudego.

W poniższym paragrafie pomijamy uwzględnienie wpływu wektora falowego na wartości $\varepsilon$ i $\mu$. W ogólności $\varepsilon(\omega,\vec{k})$ jes t funkcją zarówno częstotliwości jak i wektora falowego, co należy rozumieć jako zależność indukcji elektrycznej $\vec{D(t,\ver{r})}$ nie tylko od wzbudzenia w poprzedzającej chwili czasu $t'$, ale również od wzbudzenia fali elektromagnetycznej w otoczeniu $\vec{r}'$. Ze względu na zależność od otoczenia ta klasa zjawisk nazywana jest nielokalnymi. Wpływu otoczenia na stan polaryzacji $\vec{P}$ nie można pomijać gdy zmienność pola elektromagnetycznego jest znacząca w porównaniu z drogą swobodną elektronów w ośrodku.

\subsection{Model Lorenza-Drudego}
Własności materiałowe w tym modelu opisywane są przez

