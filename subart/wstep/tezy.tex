Głównymi zadaniami realizowanymi przez autora w~trakcie studiów dokotranckich było wykorzystanie metod obliczeniowych w~celu projektowania i~optymalizacji struktury podfalowych w~celu kształtowania fal elektormagnetyczncych. Rozważanie dotyczyły zarówno zakresu widzialnego i~dalekiej podczerwieni, aż po symulacje dotyczące fal E-M o~częstotliwościach terahercowych. W wyniku przeprowadzonych prac sformuowane zostały następujące tezy, dotyczące zarówno aspektów teoretycznych jak i~posiadających kontekst eksperymentalny:
\begin{itemize}

\item Podwójne siatki metalowe, jako układy liniowe, zgodne z~twierdzeniem o~wzajmeności Lorenza nie mogą posłużyć do konstrukcji izolatorów optycznych. Asymetria w~kierunku transmisji w~odpowiednio zaprojekotwanych strukturach DMG osiągana jest w~-1~i~+1 rzędzie dyfrkacyjnym, dzięki czemu omawiane struktury mogą zostać wykorzystane w~geometrii cylindrycznej do konstrukcji soczewek dyfrakcyjnych.

\item Możliwe jest wykorzystanie metalowych siatek dyfrakcyjnych w~celu budowy wąskopasmowych, efektywnych anten promieniowania THz opartego na trazystorach polowych realizowanych w~podkładzie z~półprzewodników. 

\item Możliwa jest eksperymentalna realizacja metamateriału absorbcyjnego o~własnościach podobnych do PML dla długości fali z~zakresu podczerwieni. Tego typu absorbery choć złożone z~warstw o~rozmiarach podfalowych, same osiągają jednak grubości zbliżone do długości fali lub większe gdy wymagane jest ośiągnięcie niskiego współczynnika odbicia dla padania pod kątami bliskimi $90^{\circ}$.

\item Realizacja metamateriałów charakteryzujących się propagacją światła nie podlegającą ograniczeniu dyfrakcyjnemu opartych o~wielowarstwy metaliczno-dielektryczne wymaga możliwości bardzogładkiego napylania warstw. Chropowatości charakteryzująces się $RMS>1$~nm mogą uniemożliwić praktyczne wykorzystanie takich struktur, szczególnie w~przypadku stosów zawierających znaczną liczbę (ok 10) warstw.
\end{itemize}
