Niniejsza rozprawa poświęcona jest strukturom fotonicznym zawierającym elementy metaliczne o rozmiarach charakterystycznych mniejszych od długości fali, w szczególności w postaci siatek, lub struktur cienkowarstwowych. Badania były ukierunkowane na uzyskanie  transmisji nadrozdzielczej i kontrolowanie jej kierunku, na uzyskanie transmisji asymetrycznej oraz jednokierunkowej, a także na kształtowanie właściwości absorpcyjnych. W wyniku przeprowadzonych prac sformułowane zostały następujące tezy:

\begin{itemize}
\item Podwójne podfalowe siatki metalowe (ang. double metallic grating - DMG) wykazują asymetrię transmisji w +1~i~-1 rzędach dyfrakcyjnych dla przeciwnych kierunków propagacji. Jednocześnie, jako układy złożone z materiałów optycznie liniowych, zgodne z twierdzeniem o wzajemności Lorenza, nie wykazują asymetrii transmisji w zerowym rzędzie i nie mogą posłużyć do konstrukcji izolatorów optycznych. W geometrii cylindrycznej można je wykorzystać do konstrukcji jednokierunkowych soczewek dyfrakcyjnych.
% Podwójne siatki metalowe, jako układy liniowe, zgodne z~twierdzeniem o~wzajemności Lorenza nie mogą posłużyć do konstrukcji izolatorów optycznych. Asymetria w~kierunku transmisji w~odpowiednio zaprojektowanych strukturach DMG~(ang. double metallic grating) osiągana jest w~-1~i~+1 rzędzie dyfrakcyjnym, dzięki czemu omawiane struktury mogą zostać wykorzystane w~geometrii cylindrycznej do konstrukcji soczewek dyfrakcyjnych.

\item Metalowe siatki dyfrakcyjne umożliwiają wydajne, widmowo selektywne, sprzęganie fali z przestrzeni swobodnej do struktur falowodowych. Można je wykorzystać jako element anteny promieniowania terahercowego dla detektorów na tranzystorach polowych.
% Możliwe jest wykorzystanie metalowych siatek dyfrakcyjnych w~celu budowy wąskopasmowych, efektywnych anten promieniowania THz opartego na tranzystorach polowych realizowanych w~podkładzie z~półprzewodników. 

\item Możliwa jest realizacja metamateriału absorpcyjnego o~własnościach przypominających PML~(ang. perfectly matched layer) dla długości fali z~zakresu podczerwieni. Tego typu absorbery choć złożone z~warstw o~rozmiarach podfalowych, same osiągają jednak grubości zbliżone do długości fali lub większe - gdy wymagane jest osiągnięcie niskiego współczynnika odbicia dla padania pod kątami bliskimi $90^{\circ}$.

\item Realizacja metamateriałów charakteryzujących się propagacją światła nie podlegającą ograniczeniu dyfrakcyjnemu opartych na~wielowarstwach metaliczno-dielektryczne wymaga bardzo gładkiego napylania warstw. Chropowatości warstw przekraczające~$1$~nm\footnote{Wartość $1$~nm odnosi się do odchlenia średniokwadratowego (ang. root mean square - RMS) od zamierzonej grubości warstwy} mogą uniemożliwić praktyczne wykorzystanie takich struktur, szczególnie w~przypadku stosów zawierających 10 bądź więcej warstw.
\end{itemize}
