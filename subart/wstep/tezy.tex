Głównymi zadaniami realizowanymi przez autora było wykorzystanie metod obliczeniowych w~celu projektowania i~optymalizacji struktury podfalowych do kształtowania fal elektromagnetycznych. Rozważania dotyczyły nie tylko zakresu widzialnego, ale również dalekiej podczerwieni, aż po symulacje dotyczące fal E-M o~częstotliwościach terahercowych. W wyniku przeprowadzonych prac sformułowane zostały następujące tezy, dotyczące aspektów teoretycznych jak i~posiadających kontekst eksperymentalny:
\begin{itemize}
\item Podwójne siatki metalowe, jako układy liniowe, zgodne z~twierdzeniem o~wzajemności Lorenza nie mogą posłużyć do konstrukcji izolatorów optycznych. Asymetria w~kierunku transmisji w~odpowiednio zaprojektowanych strukturach DMG~(ang. double metallic grating) osiągana jest w~-1~i~+1 rzędzie dyfrakcyjnym, dzięki czemu omawiane struktury mogą zostać wykorzystane w~geometrii cylindrycznej do konstrukcji soczewek dyfrakcyjnych.

\item Możliwe jest wykorzystanie metalowych siatek dyfrakcyjnych w~celu budowy wąskopasmowych, efektywnych anten promieniowania THz opartego na tranzystorach polowych realizowanych w~podkładzie z~półprzewodników. 

\item Możliwa jest eksperymentalna realizacja metamateriału absorpcyjnego o~własnościach przypominających PML~(ang. perfectly matched layer) dla długości fali z~zakresu podczerwieni. Tego typu absorbery choć złożone z~warstw o~rozmiarach podfalowych, same osiągają jednak grubości zbliżone do długości fali lub większe gdy wymagane jest osiągnięcie niskiego współczynnika odbicia dla padania pod kątami bliskimi $90^{\circ}$.

\item Realizacja metamateriałów charakteryzujących się propagacją światła nie podlegającą ograniczeniu dyfrakcyjnemu opartych o~wielowarstwy metaliczno-dielektryczne wymaga bardzo gładkiego napylania warstw. Chropowatości charakteryzujące się $RMS>1$~nm mogą uniemożliwić praktyczne wykorzystanie takich struktur, szczególnie w~przypadku stosów zawierających znaczną liczbę (około 10) warstw.
\end{itemize}
