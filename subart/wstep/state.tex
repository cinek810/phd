%Optyka jest częścią fizyki, która po szkolnym kursie kojarzy się większości z~wyznaczaniem biegu promieni świetlnych zgodnie z~zasadami optyki geometrycznej. Podstawowe elementy, z~których budowane są zadania do rozwiązania przez uczniów to soczewki i~pryzmaty, tworzące niezbyt skomplikowany, w~porównaniu z~innymi, dział fizyki. Powstaje wrażenie, że w~nauce o~świetle nie ma miejsca na niespodzianki, nowe odkrycia czy nawet zaskakujące wykorzystanie znanych praw. Zgłębianie zagadnień związanych z~elektromagnetyzmem, prowadzi nas jednak w~świat, w~którym dokonuje się wielu zaskakujących odkryć poszerzających nasze zrozumienie i~umożliwiających różnorakie zastosowania.

Skupiając się na ostatnich trzydziestu latach historii optyki dostrzec możemy wiele kroków milowych dokonanych przez fizyków na całym świecie. Z pewnością jednym z~najbardziej istotnych był opis kryształów fotonicznych, w~których struktura geometryczna narzuca ograniczenia na ruch fotonów na zasadach analogicznych do wpływu jaki wywiera sieć krystaliczna w~ciałach stałych na poruszające się w~nich elektrony. Chociaż tego rodzaju struktury były badane przez ludzi jeszcze w~XIX wieku, to sam termin jak i~nowatorskie podejście znajdujące głęboką analogię do fizyki półprzewodników pojawiły się dopiero po kluczowych pracach Eli Yablonovitscha~\cite{yablonovitch1987inhibited} i~Sajeeva Johna~\cite{john1987strong}. W szczególności przytoczeni autorzy zauważyli możliwość występowania fotonicznej przerwy wzbronionej. Poparcie wniosków teoretycznych wytworzonymi trójwymiarowymi kryształami fotonicznymi dla zakresu mikrofalowego~\cite{yablonovitch1991photonic} wraz z późniejszym wytworzeniem przez Kraussa i in. dwuwymiarowej struktury z fotoniczną przerwą wzbronioną dla długości fali E-M odpowiadającej światłu widzialnemu~\cite{krauss1996two} otworzyło drogę do zupełnie nowych zastosowań.

Znaczący wpływ na postrzeganie elektromagnetyzmu miało wprowadzenie ,,optyki transformacyjnej'', której podwaliny stworzyli Ward i~Pendry~\cite{ward1996refraction}. Zaproponowane  przez nich podejście do równań Maxwella polegające na równoważnym potraktowaniu transformacji przestrzeni i~przenikalności elektrycznej i~magnetycznej jest analogiczne do zakrzywienia przestrzeni przez grawitację w~ogólnej teorii względności. Najbardziej spektakularnym przewidywaniem teoretycznym, opartym na optyce transformacyjnej, które zostało również potwierdzone w~eksperymentach jest płaszcz niewidzialności~\cite{schurig2006metamaterial}.

%Jedynym z~podstawowych zastosowań optyki przez stulecia było obrazowanie, czyli tworzenie obrazu rzeczywistego obiektu w~innym miejscu w~przestrzeni niż znajduje się obiekt. Zastosowanie znajdują tu zazwyczaj soczewki, nieodłącznym w~dziejach ludzkości elementem optycznym wykorzystującym naturalną soczewkę, do obrazowania właśnie, jest ludzkie oko. Ewentualne wady ludzkiej soczewki mogą być korygowane za pomocą dodatkowych soczewek w~postaci okularów czy szkieł kontaktowych. Zapewne z tego względu wielu moich znajomych w~trakcie studiów, gdy mówiłem, że zajmuję się optyką kojarzyło mnie z~kimś zajmującym się okularami.

Jedynym z~podstawowych zastosowań optyki przez stulecia było obrazowanie, czyli tworzenie obrazu rzeczywistego obiektu w~innym miejscu w~przestrzeni niż znajduje się sam obiekt. Zastosowanie znajdują tu zazwyczaj soczewki. Obrazowanie za pomocą tradycyjnych elementów optycznych posiada jednak znaczące ograniczenia wynikające ze zjawiska dyfrakcji, przy ich pomocy nie jest możliwe skupianie światła w~obszarach znacznie mniejszych od połowy długości fali, funkcjonujące pod nazwą ograniczenia Rayleigha. Pierwsza propozycja teoretyczna stworzenia idealnej soczewki została podana przez Pendry'ego \cite{PhysRevLett.85.3966}, a oparta była na wykorzystaniu materiałów o ujemnej przenikalności elektrycznej, których własności teoretycznie analizował już w~latach sześćdziesiątych XX wieku Wiesiełago \cite{veselago1968electrodynamics}. Dalsze prace dotyczące supersoczewek czy hipersoczewek \cite{liu2007far} znacznie poszerzyły potencjalne zastosowania światła widzialnego w~obszarach takich jak obrazowanie czy litografia wysokorozdzielcza wykorzystująca światło widzialne. Prowadząc w ten sposób do wzrostu zainteresowania plazmoniką - dziedziną opisującą fale plazmonowe, których występowanie odpowiada za mechanizmy fizyczne wykorzystywane w realizacji wspomnianych elementów. Obrazowaniu nadrozdzielczemu poświęcony jest rozdział \ref{art:nondiff}.

Innym odkryciem dla którego kluczowe znaczenia ma występowanie powierzchniowych plazmonów polarytonów jest nadzwyczajna transmisja fal elektro-magnetycznych przez szczeliny o rozmiarach podfalowych. Analiza tego zjawiska została przedstawiona w~1998 przez Ebbesena i~innych \cite{ebbesen1998extraordinary}. Prace te stanowią podstawę dla analizowanych w~rozdziale \ref{chap:thz} niniejszej rozprawy podfalowych siatek dyfrakcyjnych wykazujących transmisję asymetryczną.

Poniższa praca jest kolejnym, małym wkładem, czynionym prez tysiące fizyków na całym świecie służącym pogłębienu wiedzy o~świecie, w~szczgólności o~zjawiskach optycznych i~umożliweniu jej zastosowania. Przedstawione wyżej odkrycia stanowią punkt wyjścia dla porblemów rozwiązywanych w poniższej pracy.
