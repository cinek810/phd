%Optyka jest częścią fizyki, która po szkolnym kursie kojarzy się większości z~wyznaczaniem biegu promieni świetlnych zgodnie z~zasadami optyki geometrycznej. Podstawowe elementy, z~których budowane są zadania do rozwiązania przez uczniów to soczewki i~pryzmaty, tworzące niezbyt skomplikowany, w~porównaniu z~innymi, dział fizyki. Powstaje wrażenie, że w~nauce o~świetle nie ma miejsca na niespodzianki, nowe odkrycia czy nawet zaskakujące wykorzystanie znanych praw. Zgłębianie zagadnień związanych z~elektromagnetyzmem, prowadzi nas jednak w~świat, w~którym dokonuje się wielu zaskakujących odkryć poszerzających nasze zrozumienie i~umożliwiających różnorakie zastosowania.

Wśród ważnych wydarzeń,~w które obfituje historia ostatnich trzydziestu lat optyki,~z pewnością wymienić należy sformułowanie teorii kryształów fotonicznych. Periodyczna struktura kryształu fotonicznego określa zachowanie fotonów, podobnie jak sieć krystaliczna w~ciałach stałych wpływa na poruszajace się w nich elektrony. Geometria kryształu fotonicznego determinuje jego strukturę modową, właściwości dyspersyjne, a co za tym idzie właściwości refrakcyjne i dyfrakcyjne, a w przypadku kryształów plazmonicznych także absorpcyjne. Chociaż tego rodzaju struktury były badane przez ludzi jeszcze w~XIX wieku, zarówno sam termin jak i~nowatorskie podejście znajdujące głęboką analogię do fizyki półprzewodników pojawiły się dopiero po kluczowych pracach Eli Yablonovitscha~\cite{yablonovitch1987inhibited} i~Sajeeva Johna~\cite{john1987strong}. W szczególności przytoczeni autorzy zauważyli możliwość występowania fotonicznej przerwy wzbronionej. Poparcie wniosków teoretycznych wytworzonymi trójwymiarowymi kryształami fotonicznymi dla zakresu mikrofalowego~\cite{yablonovitch1991photonic} wraz z późniejszym wytworzeniem przez Kraussa i in. dwuwymiarowej struktury z fotoniczną przerwą wzbronioną dla długości fali elektromagnetycznej odpowiadającej światłu widzialnemu~\cite{krauss1996two} otworzyło drogę do zupełnie nowych zastosowań.

Znaczący wpływ na postrzeganie elektromagnetyzmu miało wprowadzenie optyki transformacyjnej~\cite{ward1996refraction,han2008ray,kildishev2008engineering,li2009designing,gao2016translation}, której podwaliny stworzyli Ward i~Pendry. Zaproponowane  przez nich podejście do równań Maxwella, polegające na równoważnym potraktowaniu transformacji przestrzeni i~przenikalności elektrycznej i~magnetycznej, jest analogiczne do opisu zakrzywienia przestrzeni przez grawitację w~ogólnej teorii względności. Najbardziej spektakularnym przewidywaniem teoretycznym, opartym na optyce transformacyjnej, które zostało również potwierdzone w~eksperymentach, jest płaszcz niewidzialności~(ang. electromagnetic cloak)~\cite{schurig2006metamaterial,sun2016optic,alitalo2009electromagnetic}.

%Jedynym z~podstawowych zastosowań optyki przez stulecia było obrazowanie, czyli tworzenie obrazu rzeczywistego obiektu w~innym miejscu w~przestrzeni niż znajduje się obiekt. Zastosowanie znajdują tu zazwyczaj soczewki, nieodłącznym w~dziejach ludzkości elementem optycznym wykorzystującym naturalną soczewkę, do obrazowania właśnie, jest ludzkie oko. Ewentualne wady ludzkiej soczewki mogą być korygowane za pomocą dodatkowych soczewek w~postaci okularów czy szkieł kontaktowych. Zapewne z tego względu wielu moich znajomych w~trakcie studiów, gdy mówiłem, że zajmuję się optyką kojarzyło mnie z~kimś zajmującym się okularami.

Jednym z~podstawowych zastosowań optyki przez stulecia było obrazowanie, czyli tworzenie obrazu rzeczywistego obiektu w~innym miejscu w~przestrzeni. Zastosowanie znajdują tu zazwyczaj soczewki. Obrazowanie za pomocą tradycyjnych elementów optycznych ma jednak znaczące ograniczenia wynikające ze zjawiska dyfrakcji. Przy ich pomocy nie jest możliwe skupianie światła w~obszarach znacznie mniejszych od połowy długości fali, co wynika z ograniczenia Rayleigha. Pierwsza teoretyczna propozycja realizacji idealnej soczewki została podana przez Pendry'ego \cite{PhysRevLett.85.3966,loschialpo2003electromagnetic,smith2003limitations,ramakrishna2002asymmetric}, a oparta była na wykorzystaniu materiałów o ujemnej przenikalności elektrycznej~(Pendry zauważył, że dla pola bliskiego dla wybranej polaryzacji można zaniedbać właściwości magnetyczne), których własności teoretycznie analizował już w~latach sześćdziesiątych XX wieku Wiesiełago~(ang. transkrypcja Veselago)~\cite{veselago1968electrodynamics}. Dalsze prace dotyczące supersoczewek czy hipersoczewek \cite{liu2007far,jacob2006optical,jacob2007semiclassical,ma2010advances,rho2010spherical} znacznie poszerzyły potencjalne zastosowania światła widzialnego w~obszarach takich jak obrazowanie czy litografia wysokorozdzielcza wykorzystująca światło widzialne. Prowadząc w ten sposób do wzrostu zainteresowania plazmoniką - dziedziną opisującą fale plazmonowe, których występowanie odpowiada za mechanizmy fizyczne wykorzystywane w realizacji wspomnianych elementów. Obrazowaniu nadrozdzielczemu poświęcony jest rozdział \ref{art:nondiff}.
 
Innym odkryciem, dla którego kluczowe znaczenie ma występowanie powierzchniowych plazmonów polarytonów, jest nadzwyczajna transmisja fal elektromagnetycznych przez szczeliny o rozmiarach podfalowych. Analiza tego zjawiska została przedstawiona w~1998 przez Ebbesena i~innych \cite{ebbesen1998extraordinary}. Prace te stanowią podstawę dla analizowanych w~rozdziale \ref{chap:thz} niniejszej rozprawy podfalowych siatek dyfrakcyjnych wykazujących transmisję asymetryczną.

%Poniższa praca jest kolejnym, małym wkładem, czynionym prez tysiące fizyków na całym świecie służącym pogłębienu wiedzy o~świecie, w~szczgólności o~zjawiskach optycznych i~umożliweniu jej zastosowania. Przedstawione wyżej odkrycia stanowią punkt wyjścia dla porblemów rozwiązywanych w poniższej pracy.
