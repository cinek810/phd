Optyka jest częścią fizyki, która po szkolnym kursie kojarzy się większości z~wyznaczaniem biegu promieni świetlnych zgodnie z~zasadami optyki geometrycznej. Podstawowe elementy z~których budowane są zadania do rozwiązania przez uczniów to soczewki i~pryzmaty, tworzące niezbyt skomplikowany, w~porównaniu z~innymi dział fizyki. Powstaje wrażenie, że w~nauce oświetle nie ma miejsca na niespodzianki, nowe odkrycia czy nawet zaskakujące wykorzystanie znanych praw. Zgłębianie zagadnień zwiazanych z~elektromagnetyzmem, prowadzi nas jednak w~świat w~którym dokonuje się wielu zaskakujących odkryć poszerzających nasze zrozumienie i~umożliwiających różnorakie zastosowania.

Skupiając się na ostatnich trzydziestu latach historii optyki dostrzec możemy wiele kroków milowych dokonanych przez fizyków na całym świecie. Z pewnością jednym z~najbardziej istotnych był opis kryształów fotonicznych, w~których struktura geometryczna narzuca ograniczenia na ruch fotonów na zasadach analogicznych do wpływu jaki wywiera sieć krystaliczna w~ciałach stałych na poruszające się w~nich elektrony. Chociaż struktury wykazujące omawiane własności były badane przez ludzi jeszcze w~XIX wieku, to sam termin jak i~nowatorskie podejście znajdujące głęboką analogię do fizyki półprzewodników pojawiły się dopiero po kluczowych pracach Eli Yablonovitscha~\cite{yablonovitch1987inhibited} i~Sajeeva Johna~\cite{john1987strong}.

Znaczący wpływ na postrzeganie elektromagnetyzmu miało wprowadzenie ,,optyki transformacyjnej'', której podwaliny stworzyli Ward i~Pendry~\cite{ward1996refraction}. Zaproponowane  przez nich podejście do równań Maxwella polegające na równoważnym potrafkowaniu transofmracji przestrzeni i~przenikalności elektrycznej i~magnetycznej jest analogiczne do zakrzywienie przestrzeni przez grawitację w~ogólnej teorii względności. Najbardziej spektakularnym przewidywaniem teoretycznym, opartym na optyce transformacyjnej, które zostało również potwierdzone w~eksperymentach jest płaszcz niewidzialności~\cite{schurig2006metamaterial}.

Jedynym z~najbardziej podstawowych zastosowań optyki przez stulecia było obrazowanie, czyli tworzenie obrazu rzeczywistego obiektu w~innym miejscu w~przestrzeni niż znajduje się obiekt. Zastosowanie znajdują tu zazwyczaj soczewki, nieodłącznym w~dziejach ludzkości elementem optycznym wykorzystującym naturalną soczewkę do obrazowania właśnie jest ludzkie oko. Ewentualne wady ludzkiej soczewki mogą być korygowane przy pomocy dodatkowych soczewek w~postaci okluarów czy szkieł kontaktowych, ze względu na co wielu moich znajomych w~trakcie studiów, gdy mówiłem, że zajmuje się optyką kojarzyło mnie z~kimś zajmującym się okularami.

Tradycyjne soczewki posiadają jednak znaczące ograniczenie (nieistotne w~kontakście konstrukcji okularów do korekcji widzenia). Przy ich pomocy nie jest możliwe skupianie światła w~obszarach znacznie mniejszych od połowy długości fali. Pierwsza propozycja teoretyczna stworzenia idealnej soczewki została podana przez Pendry'ego \cite{PhysRevLett.85.3966} oparta była na wykorzystaniu materiałów oujemnej przenikalności elektrycznej, których własności teoretycznie analizował już w~latach '60 XX wieku Wiesiełago \cite{veselago1968electrodynamics}. Dalsze prace dotyczące supersoczewek czy hipersoczewek \cite{liu2007far} znacznie poszerzyły potencjalne zastosowania światła widzialnego w~zastosowaniach takich jak  obrazowanie czy litografia wysokorozdzielcza wykorzystująca światło widzialne. Tematyce związanej z~tą częścią rozwoju optyki poświęcony jest rozdział \ref{art:nondiff}.

Innym odkryciem, podobnie jak wspomniane powyżej, dla którego kluczowe znaczenia ma występowanie powierzchniowych plazmonów polarytonów jest nadzwyczajna transmisja fal elektro-magnetycznych przez szczeliny orozmiarach podfalowych. Analiza tego zjawiska została przedstawiona w~1998 przez Ebbesena i~innych \cite{ebbesen1998extraordinary,1998APS..MAR.S1511E}. Prace te stanowią podstawę dla analizowanych w~rozdziale \ref{chap:thz} niniejszej rozprawy.

Poniższa praca jest kolejnym, małym wkładem, czynionym prez tysiące fizyków na całym świecie służącym pogłębienu wiedzy oświecie, w~szczgólności o~zjawiskach optycznych i~umożliweniu jej zastosowania.
