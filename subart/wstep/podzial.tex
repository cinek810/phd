Poniższa rozprawa doktorska składa się ze wstępu, pięciu rozdziałów  będących przedmiotem rozprawy, oraz szóstego rozdziału stanowiącego podsumowanie pracy. Wstęp zawiera niezbędne elementy wprowadzające do tematyki kształtowania fal elektromagnetycznych, oraz zawiera tezy rozprawy doktorskiej. Rozdział drugi stanowi wprowadzenie literaturowe do tematyki. Opisane w~nim zostały stosowane w~symulacjach metody numeryczne, w~szczególności dokładnie metoda różnic szkończonych w~dziedzinie czasu (ang. FDTD finite-difference time-domain). Rozdział zawiera również niezbędne informacje na temat układów liniowych niezmienniczych ze względu na przesunięcia, wraz z~elementami Optyki Furierowskiej. Wprowadzenie zawiera również wprowadzenie przybliżenia ośrodka efektywnego stosowanego dla wielowarstw dyskutowanych w~rozdziale czwartym i~piątym.

Rozdział trzeci poświęcony jest projektowaniu układów opartych o~metale i~półprzewodniki, przeznaczonych dla zakresu terahercowego. Rozdział traktuje na temat możliwości wykorzystania metalowych siatek dyfrakcyjnych w~roli anten promieniowania THz. Wskazane zostały możliwości kierowania promieniowaniem E-M, oraz transmisji  selektywnej - rezonansowej. Zachodzące zjawiska zostały przedstawione w~sposób jakościowy, w~oparciu o~współczesną literaturę przedmiotu. Dokładny opis ilościowy oparty jest o~autorskie obliczenia autora pracy, pozwalające na dokładne określenie granic stosowalności przybliżeń teoretycznych w~zakresie modelowania własności materiałów jak i~samych siatek dyfrakcyjnych. W dalszej części rozdziału przedstawione zostały struktury określane w~literaturze mianem podwójnych siatek metalowych. Opisany został wpływ odpowiednich parametrów geometryczncych DMG na współczynniki transmisji. Przedstawione zostały możliwości uzyskania transmisji asymetrycznej, wraz z~zaprzeczeniem funkcjonowania omawianych układów w~charakterze diody optycznej. Zakończenie rozdziału stanowi omówienie możliwości wykorzystania DMG w~geometrii cylindrycznej jako jednokierunkowej soczewki promieniowania THz.

Rozdziały czwarty i~piąty poświęcone są metamateriałom opartym o~struktury warstwowe. W rozdziale czwartym wyprowadzony zostaje w~oparciu o~zasady optyki transformacyjnej nieodbijający ośrodek pochłaniający promieniowanie E-M, w~obliczeniach numerycznych określany jako PML (od ang. perfectly matched layer). Omówiona zostaje możliwośc realizacji metamateriału absorbcyjnego o~charakterystyce PML przy pomocy wielowarstwy, w~granicy homogenizacji opisywanej efektywnymi tensorami przenikalności elektrycznej i~magnetycznej odpowiadającym wybranemu PML. Dyskusji poddana zostaje możliwośc realizacji metamateriału przy pomocy realnych substancji występujących w~przyrodzie, nie posiadających zysku optycznego oraz własności magnetycznych w~zakresach światła widzialnego i~podczerwieni. Ostatecznie przedstawione są wyniki eksperymentów numeryczncych dla metamateriału o~własnościach podobnych do PML opartego o~rzeczywiste substancje.

W kolejnym rozdziale omówione zostały wielowarstwy metaliczno-dielektryczne umóżliwiające propagację światła poza limitem klasycznego ograniczenia dyfrakcyjnego. W pierwszych częściach rozdziału czytelnik może zapoznać się z~współczesnym stanem wiedzy w~dziedzinie obrazowania podfalowego. W dalszej części rozdziału w~oparciu o~wyniki obliczeniowe autora pracy, oraz informacje literaturowe omówiony zostały możliwości budowy bardziej skomplikowanych elementów optycznych pozwalających na uzyskanie podfalowej koncentracji światła, oraz realizacji operacji geometrycznych rzutowania, obrotu na podfalowych rozkładach pola elektromagnetycznego.

Ostatni rozdział stanowi podsumowanie ninejszej rozprawy doktorskiej. Wskazane w~nim są kluczowe wnioski, oraz w~sposób syntetyczny wraz z~odniesieniem do treści pracy podana zostaje argumentacja tez rozprawy. 
