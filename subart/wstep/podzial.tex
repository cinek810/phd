Poniższa rozprawa doktorska składa się ze wstępu, pięciu rozdziałów  będących przedmiotem rozprawy, oraz szóstego rozdziału stanowiącego podsumowanie pracy. Wstęp zawiera niezbędne elementy wprowadzające do tematyki kształtowania fal elektromagnetycznych, oraz zawiera tezy rozprawy doktorskiej. Rozdział drugi stanowi wprowadzenie literaturowe do tematyki. Opisane w~nim zostały stosowane w~symulacjach metody numeryczne, w~szczególności dokładnie metoda różnic szkończonych w~dziedzinie czasu (ang. FDTD finite-difference time-domain). Rozdział zawiera również niezbędne informacje na temat układów liniowych niezmienniczych ze względu na przesunięcia, wraz z~elementami Optyki Fourierowskiej. Wprowadzenie zawiera również zwięzły wstęp do przybliżenia ośrodka efektywnego stosowanego dla wielowarstw dyskutowanych w~rozdziale czwartym i~piątym.

Rozdział trzeci poświęcony jest projektowaniu układów opartych na metalach i~półprzewodnikach, przeznaczonych dla zakresu terahercowego. Rozdział traktuje na temat możliwości wykorzystania metalowych siatek dyfrakcyjnych w~roli anten promieniowania THz. Wskazane zostały możliwości kierowania promieniowaniem E-M, oraz transmisji  selektywnej - rezonansowej. Zachodzące zjawiska zostały przedstawione w~sposób jakościowy, w~oparciu o~współczesną literaturę przedmiotu. Dokładny opis ilościowy oparty jest o~autorskie obliczenia autora pracy, pozwalające na dokładne określenie granic stosowalności przybliżeń teoretycznych w~zakresie modelowania własności materiałów jak i~samych siatek dyfrakcyjnych. W dalszej części rozdziału przedstawione zostały struktury określane w~literaturze mianem podwójnych siatek metalowych. Opisany został wpływ odpowiednich parametrów geometryczncych DMG na współczynniki transmisji. Przedstawione zostały możliwości uzyskania transmisji asymetrycznej, wraz z~zaprzeczeniem funkcjonowania omawianych układów w~charakterze diody optycznej. Zakończenie rozdziału stanowi omówienie możliwości wykorzystania DMG w~geometrii cylindrycznej jako jednokierunkowej soczewki promieniowania THz.

Rozdziały czwarty i~piąty poświęcone są metamateriałom opartym na strukturach warstwowych. W rozdziale czwartym wyprowadzony zostaje w~oparciu o~zasady optyki transformacyjnej nieodbijający ośrodek pochłaniający promieniowanie E-M, w~obliczeniach numerycznych określany jako PML (od ang. perfectly matched layer). Omówiona zostaje możliwośc realizacji metamateriału absorpcyjnego o~charakterystyce PML przy pomocy wielowarstwy, w~granicy homogenizacji opisywanej efektywnymi tensorami przenikalności elektrycznej i~magnetycznej odpowiadającymi wybranemu PML. Dyskusji poddana zostaje możliwośc realizacji metamateriału przy pomocy substancji występujących w~przyrodzie, nie posiadających zysku optycznego oraz własności magnetycznych w~zakresach światła widzialnego i~podczerwieni. Ostatecznie przedstawione są wyniki eksperymentów numeryczncych dla metamateriału o~własnościach podobnych do PML składającego się z~rzeczywistych substancje.

W kolejnym rozdziale omówione zostały wielowarstwy metaliczno-dielektryczne umożliwiające propagację światła niepodlegającą klasycznemu ograniczeniu dyfrakcyjnemu. W pierwszych częściach rozdziału czytelnik może zapoznać się ze współczesnym stanem wiedzy w~dziedzinie obrazowania podfalowego. W dalszej części rozdziału, w~oparciu o~wyniki obliczeniowe autora pracy, oraz informacje literaturowe omówione zostały możliwości budowy bardziej skomplikowanych elementów optycznych pozwalających na uzyskanie podfalowej koncentracji światła, oraz na realizację operacji geometrycznych rzutowania, obrotu na podfalowych rozkładach pola elektromagnetycznego.

Ostatni rozdział stanowi podsumowanie rozprawy. Wskazane w~nim są kluczowe wnioski, podana zostaje argumentacja tez rozprawy, wraz z odniesieniem do treści pracy.
