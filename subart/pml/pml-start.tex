Absorbery elektromagnetyczne znajdują zastosowania m. in. w budowie detektorów, fotowoltaice oraz kolorowaniu plazmonicznym. Niniejszy rozdział zawiera propozycję wykorzystania teorii ośrodków efektywnych do projektowania absorberów o~konstrukcji warstwowej. W oparciu o podobną postać tensora przenikalności elektrycznej materiału UPML~(ang. uniaxial perfectly matched layer; por. rozdział~\ref{art:pml}) i efektywnego tensora przenikalności elektrycznej struktury warstwowej~(patrz podrozdział \ref{subart:effmedium}) możliwe jest, w ograniczonym stopniu, odzwierciedlenie własności materiału UPML za pomocą struktury warstwowej. Początek rozdziału stanowi wprowadzenie do tematyki absorberów. W~dalszej części przedstawione zostało wyprowadzenie absorbera numerycznego UPML za pomocą optyki transformacyjnej~\cite{pendry2012transformation}. W kolejnym podrozdziale zaprezentowano możliwość realizacji metamateriału o~własnościach efektywnych odpowiadających warstwie UPML za pomocą wielowarstwy~\cite{ania2015}. Zaproponowano również wielowarstwę opartą o~dostępne materiały, wykazującą własności nieodbijającej warstwy absorpcyjnej dla długości fali $8$~µm.


