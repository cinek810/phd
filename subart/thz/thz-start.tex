Poniższy rozdział dotyczy wykorzystania siatek dyfrakcyjnych jako elementów detektorów promieniowania, umożliwiających selektywną transmisję oraz skierowanie promieniowania do tranzystora polowego stanowiącego faktyczny detektor. W dalszej części rozdziału przedstawione są możliwości wykorzystania podwójnych metalowych siatek dyfrakcyjnych do uzyskania transmisji jednokierunkowej, które w geometrii cylindrycznej mogą być również wykorzystane do koncentracji fali E-M. Poza zakresem poniższej pracy znajdują się zjawiska fizyczne związane z generacją i detekcją promieniowania THz.

Do modelowania promieniowania elektromagnetycznego w zakresie THz używane są metody tradycyjnie wykorzystywane w optyce, w  szczególności metoda FDTD. Ze względu na różnicę w długości fali, własności materiałów w zakresie THz znacznie różnią się od tych dla światła widzialnego. Kluczowymi procesami odpowiedzialnymi za wartość przenikalności elektrycznej ciał stałych dla niskich częstotliwości THz, określanych niekiedy jako sub teraherzowe, są mechanizm Drudego (patrz sekcja \ref{subart:lorenz-drude}) i relaksacja Debye'a. Dla częstotliwości bliższych dalekiej podczerwieni podstawowe znaczenie mają optyczne fonony - skwantowane mody drgań sieci krystalicznej. W zależności od wykorzystywanych materiałów typowe wartości współczynnika załamania dla polimerów znajdują się w przedziale $n \in (1.4;1.5)$, dla półprzewodników $n\in (3.1;3.5)$ i charakteryzują się niewielką dyspersją. Wypolerowane powierzchnie metalowe są wykorzystywane jako zwierciadła o współczynniku odbicia $R\approx 0.99$. \cite{lee2009principles}


