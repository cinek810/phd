Niniejszy rozdział dotyczy wykorzystania siatek dyfrakcyjnych jako elementów detektorów promieniowania, umożliwiających transmisję selektywną ze względu na częstotliwość oraz skierowanie promieniowania do tranzystora polowego stanowiącego faktyczny detektor. W szczególności w~podrozdziale \ref{subart:rezo-grating} omówione są możliwości wykorzystania metalowych siatek dyfrakcyjnych do uzyskania transmisji rezonansowej w~zakresie THz. Następnie w~części \ref{subart:antenaThz} zaprojektowane zostały siatki dyfrakcyjne do wzbudzenia modu falowodowego umożliwiającego transmisję promieniowana E-M w~kierunku detektora.

W dalszej części rozdziału przedstawione są możliwości wykorzystania podwójnych metalowych siatek dyfrakcyjnych do uzyskania transmisji jednokierunkowej. W geometrii cylindrycznej, tego rodzaju siatki, mogą być również wykorzystane do koncentracji fali E-M. Poza zakresem poniższej pracy znajdują się zjawiska fizyczne związane z~generacją i~detekcją promieniowania THz.

Do modelowania promieniowania elektromagnetycznego w~zakresie THz używane są metody tradycyjnie wykorzystywane w~optyce, w~ szczególności metoda FDTD. Ze względu na różnicę w~długości fali, własności materiałów w~zakresie THz znacznie różnią się od tych dla światła widzialnego. Różnicom tym poświęcony jest podrozdział \ref{subart:thzmat}.
