Niniejszy rozdział poświęcony jest metalowym siatkom dyfrakcyjnym o okresie bliskim lub mniejszym od długości fali, a~także siatkom o~bardziej złożonej budowie, których okres jest mniejszy od długości fali z~jednej strony, a~większy z drugiej. Podstawowe rozróżnienie pomiędzy zwykłymi, a podfalowymi siatkami wiąże się z tym, że siatki podfalowe mają jedynie zerowy rząd ugięcia w~transmisji i~w~odbiciu, czyli w~zasadzie nie są elementami dyfrakcyjnymi, choć w~niniejszej pracy dalej będą określane mianem siatek dyfrakcyjnych. Siatki o~różnym okresie z~obu stron są nowego rodzaju elementem dyfrakcyjnym, który może charakteryzować się silnie asymetryczną, bądź jednokierunkową transmisją. Najważniejsze wyniki przedstawione w~rozdziale dotyczą tego typu siatek.

Siatki metalowe stanowią ważny element optyki terahercowej~\cite{zhang2010introduction}. Pierwsza część rozdziału dotycząca ich użycia jako selektywnych widmowo sprzęgaczy do struktur falowodowych motywowana jest możliwoscią użycia takich struktur w~detektorach promieniowania THz opartych na tranzystorach polowych. Również w~przypadku struktur o~transmisji asymetrycznej, ich realizacja doświadczalna została przeprowadzona w~zakresie terahercowym~\footnote{Pomiary właściwości transmisyjnych siatek w zakresie terahercowym zostały wykonane w zakładzie fizyki ciała stałego na Wydziale Fizyki UW przez Dymitra Jaworskiego (Dmitriy Yavorskiy) oraz dra hab. Jerzego Łusakowskiego. Autor rozprawy uczestniczył w projektowaniu eksperymentu i opracowaniu wyników. Prace dotyczące detektorów promieniowania THz prowadzone były przez grupę dra hab. Jerzego Łusakowskiego.}. W~podrozdziale \ref{subart:rezo-grating} omówione są możliwości wykorzystania metalowych siatek dyfrakcyjnych do uzyskania transmisji rezonansowej w~zakresie THz. Następnie w~części \ref{subart:antenaThz} zaprojektowane zostały siatki dyfrakcyjne do wzbudzenia modu falowodowego umożliwiającego transmisję promieniowania E-M w~kierunku detektora.

W dalszej części rozdziału przedstawione są możliwości wykorzystania podwójnych metalowych siatek dyfrakcyjnych do uzyskania transmisji jednokierunkowej. W geometrii cylindrycznej, tego rodzaju siatki, mogą być również wykorzystane do koncentracji fali E-M. Poza zakresem niniejszej pracy znajdują się zjawiska fizyczne związane z~generacją i~detekcją promieniowania THz.

Do modelowania promieniowania elektromagnetycznego w~zakresie THz używane są metody tradycyjnie wykorzystywane w~optyce, w~ szczególności metoda FDTD. Ze względu na różnicę w~długości fali, własności materiałów w~zakresie THz znacznie różnią się od tych dla światła widzialnego. Różnicom tym poświęcony jest podrozdział \ref{subart:thzmat}.
