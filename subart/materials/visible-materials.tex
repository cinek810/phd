\subsection{Model Lorenza-Drudego}
\label{subart:lorenz-drude}
Powszechnie wykorzystywanym do opisu własności dyspersyjnych materiałów jest tzw. model Lorenza-Drudego. \textit{De facto} jest on połączeniem opisu substancji przewodzących (opisywanych modelem Drudego), oraz dielektryków (opisywanych modelem Lorenza). Wyprowadzenie obu modeli opiera się na zastosowaniu zasad mechaniki klasycznej w stosunku do cząstek naładowanych znajdujących się w materii. Dla dielektryków przyjmujemy, że są one silnie związane z węzłami sieci krystalicznej, a ruch każdego ładunku opisuje równanie oscylatora tłumionego, pobudzanego siłą harmoniczną wywoływaną przez zewnętrzne pole elektromagnetyczne:
\begin{equation}
m \frac{d^2 \vec{r}}{dt^2} + m \gamma \frac{d \vec{r}}{dt} + m \omega^2_1 \vec{r} = q \vec{E_0} e^{i \omega t}.
\label{eq:newton-lorenz}
\end{equation}

W powyższym równaniu $m$ nie należy traktować jako masy ładunku, a jako parametr stanowiący tzw. masę efektywną, której wartość można wyznaczyć za pomocą mechaniki kwantowej. Paramtetry $\omega_1$ i $\gamma$ możemy zgodnie z mechaniką klasyczną interpretować odpowiednio jako częstość własną i wpółczynnik tłumienia oscylatora. W przypadku substancji o różnych węzłach sieci krystalicznej, równanie \ref{eq:newton-lorenz} należy zapisać osobno dla każdego rodzaju ładunków i centrów sieci. Rozwiązując powyższe równanie różniczkowe, oraz korzystając z definicji polaryzacji możemy wyznaczyć przenikalność dielektryczną ośrodka nieprzewodzącego jako:

\begin{equation}
\varepsilon = 1 + \frac{N_o q^2}{\varepsilon_0 m} \sum_j \frac{f_j}{\omega_j^2 - \omega - i \gamma_j}.
\label{eq:lorenz}
\end{equation}

Wprowadzone w powyższym równaniu współczynniki $f_j$ opsiują tzw. siłę oscylatora, związaną z prawdopodobieństwem przejść między stanami częstek opisywanego materiału. Wprowadzony parametr $N_o$ opisuje koncentrację oscylatorów.  W przeciwieństwie do izolatorów elektrycznych opis metali musi przedewszystkim uwzględnić istnienie nośników swobodnych. Zaniedbując oddziaływanie ładunków swobodnych ze sobą i zakładając, że ich zderzenia z wierzchołkami sieci krystalicznej mają charakter w pełni przypadkowy możemy ich klasyczne równanie ruchu zapisać jako:

\begin{equation}
m \frac{d^2\vec{r}}{dt} + m \gamma \frac{d\vec{r}}{dt} = q \vec{E_0}e^{i\omega t}.
\label{eq:newton-drude}
\end{equation}

W powyższym równaniu wprowadzona siła oporu $m \gamma \vec{v}$ wynika ze zderzeń z węzłmi sieci krystalicznej. Rozwiązanie powyższego równanie prowadzi do następującego wyrażenia na przenikalność dielektryczną

\begin{equation}
\varepsilon= 1 - \frac{\omega_p^2}{\omega^2+i\omega \gamma},
\label{eq:drude}
\end{equation}

w którym wprowadzona wartość $\omega_p$ to częstość plazmowa, opisywana wzorem:

\begin{equation}
\omega_p = \sqrt{\frac{N q^2}{\epsilon_0 m}},
\label{eq:omega-plazmowa}
\end{equation}
gdzie $N$ jest koncentracją swobodnych nośników o ładunku $q$. Im większa koncentracja nośników swobodnych, tym większa jest częstość plazmowa opisywanego metalu.  Dla częstotliowości z zakresu optycznego $\gamma<<\omega$ co jest odzwierciedleniem faktu, że droga swobodna elektronów w paśmie przewodnictwa jest znacznie większa od rozważanych długości fali. Zgodnie z równaniem  \ref{eq:drude} oznacza to, że dla światła widzialnego decydującą rolę dla własności metali ma część rzeczywista $\varepsilon$, która jest dodatnia tylko dla $\omega>\omega_p$. Dla takich częstotliwoci w równanie falowe w metalach będzie mieć rozwiązanie w postaci fal poprzecznych. Fizyczą interpretację $\omega_p$ znaleźć można w rozwiazaniu równania \ref{eq:newton-drude}. Jest to częstość własna drgań podłużnych elektronów swobodnych. Czętość plazmowa jest natomiast podstawą dla wprowadzenia pojęcia plazmonu objętościowego, stanowiącego kwant omawianych drgań.

Ponieważ polaryzacja jest sumą efektywnych momentów diplowych przypadających na jednostkę objętości, to wkłady do polaryzacji pochodzące od odziaływań z elektronami w paśmie przewodictwa i jonami sieci krystalicznej podlagają dodawaniu. Dlatego model przenikalności elektrycznej materiałów uwzględniający oba te zjawiska, jest sumą wkładów pochodzących z równań \ref{eq:lorenz} i \ref{eq:drude}. Zazwyczaj model materiałowy dopasowywane jest do danych eksperymentalnych jedynie w ograniczonym zakresie, ze względu na to większość rezonansów z wzoru \ref{eq:lorenz} może zostać zastąpionych stałą wartością zwyczajowo określaną jako $\varepsilon_\infty$, a ostateczny wzór przyjmuje potać
\begin{equation}
\varepsilon(\omega)=\varepsilon_\infty- \frac{\omega_p^2}{\omega^2+i\omega\gamma} +\frac{Nq^2}{\varepsilon_0 m} \Sigma_j \frac{f_j}{\omega_j^2-\omega^2-i\gamma_j\omega}
\label{eq:lorenz-drude}
\end{equation}
