
Większość symulacji numerycznych wykonanych przez autora pracy, w szczególności symulacji metodą FDTD wymaga zastosowania komputerów dużej mocy~(ang HPC - High Performance Computing). Zwiększenie mocy obliczeniowej przy wykorzystaniu takich zasobów wymaga zrównoleglenia obliczeń, możliwe do uzyskania przyspieszenie w podstawowy sposób opisuje prawo Amdahla
\begin{equation}
	S=\frac{1}{(1-P)+\frac{P}{s}},
	\label{eq:amdahl}
\end{equation}
gdzie przez $S$ oznaczono całkowite przyspieszenie obliczeń, $P$ to część obliczeń która może zostać zrównoleglnoa, a $s$ to możliwe przyspieszenie części $P$ w wyniku zrównoleglenia. Przykładowe wartości przyspieszenia symulacji przy wykorzystaniu większej liczby rdzeni obliczeniowych prezentuje wykres na rysunku~\ref{fig:amdhal}.

\begin{figure}[h]
	\includegraphics[width=\textwidth]{images/amdahl.png}
	\caption{Możliwe przyspieszenie symulacji zgodnie z (\ref{eq:amdahl}), przy zerowym koszcie zrównoleglenia ($s$ równe liczbie używanych rdzeni) dla różnych wartości $P$~\cite{wiki:amdhal}}
	\label{fig:amdhal}
\end{figure}

W przypadku metod równoległego rozwiązywania równań różniczkowych, takich jak FDTD, zrównoleglenie wymaga wprowadzenia dodatkowego kroku zapewniającego uspójnienie warunków na częściach siatki stanowiących granicę między obszarami symulowanymi równolegle na różnych procesorach. Zrównoleglenie symulacji wymaga wykonania dodatkowego uspójnieniu warunków granicznych pomiędzy obszarami symulowanymi na różnych rdzeniach. Czas trwania tego dodatkowego kroku zależy przedewszystkim od opóźnień w komunikacji pomiędzy rdzeniami obliczeniowymi. Na całkowite opóźnienie składają się część aplikacyjna - zazwyczaj implementowana przy pomocy protokołu MPI~(ang. Message Passing Interface), oraz część sprzętowa. Obie te część należy w znaczącym stopniu dostosować do wykorzystywanej symulacji, o ile część aplikacyjna wymaga zastosowania odpowiednich parametrów blokowania oraz odciązania CPU z częśto wykorzystywanych elementów\footnote{Źle dobrana implementacja lub niewłaściwa konfiguracja mogą doprowadzić do znaczącego obciążenia procesora poprzez część aplikacji odpowiadającą za komunikację}. To część sprzętowa w znaczący sposób zależy od architektury połączeń między procesorami wewnątrz węzłów obliczeniowych, oraz interkonektu pomiędzy węzłami obliczeniowymi. 

Jądrem obliczeń numerycznych prowadzonych w niniejsze pracy była aplikacja meep~\cite{OskooiRo10} umożliwiająca wykonywanie obliczeń metodą FDTD.

Są to zagadnienia nie związane bezpośrednio z tematem niniejszej rozprawy. Jednak bez głębokiej wiedzy, oraz doświadczenia w programowaniu równoległym oraz prowadzeniu obliczeń na zasobach HPC uzyskanie prezentowanych wyników nie byłoby możliwe.
