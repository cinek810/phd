\label{subart:effmedium}
Przybliżenie ośrodka efektywnego (ang.~EMA effective medium approximations lub ang.~EMT effective medium theory) to określenie używane w~odniesieniu do analitycznych modeli opisujących makroskopowe własności elektromagnetyczne podfalowej struktury złożonej z~różnych materiałów. EMA pozwala opisywać niejednorodny obszar w~przestrzeni złożony z~wielu materiałów jako jeden homogeniczny obszar o~innych właściwościach. Kluczowym w~wyprowadzeniu przybliżenia EMT jest zdefiniowanie geometrii w~jakiej układane są materiały składowe. Na jej podstawie w~zależności od rozmiarów wyprowadzane są ostateczne formuły na składowe tensorów przenikalności elektrycznej $\varepsilon$ i~magnetycznej µ.

Dla niniejszej pracy szczególne znaczenie mają układy jednowymiarowe, w~których periodycznie umieszczone są kolejne warstwy ośrodków materialnych, gdy możemy zakładać, że pojedyncza warstwa jest tak cienka w~porównaniu z~długością fali, że wartości pól $E$ i~$D$ wewnątrz warstwy w~określonej chwili czasu są stałe. Możemy wyprowadzić przybliżone wartości tensora przenikalności elektrycznej:

\[ \varepsilon= \left[ \begin{array}{ccc}
	\varepsilon_{\parallel} & 0 & 0 \\
	0 & \varepsilon_{\parallel} & 0 \\
	0 & 0 &  \varepsilon_{\perp} \end{array} \right] 
\\
\]
\begin{equation}
	\begin{gathered}
		\varepsilon_{\parallel}=f\cdot{\varepsilon_1}+(1-f)\cdot \varepsilon_2 \\ 
		\varepsilon_{\perp}=\left(f\cdot{\varepsilon_1^{-1}}+(1-f)\cdot \varepsilon_2^{-1}\right)^{-1},
	\end{gathered}
\label{eq:effmedium}
\end{equation}


oraz magnetycznej:
\[ \mu= \left[ \begin{array}{ccc}
					\mu_{\parallel} & 0 & 0 \\
					0 & \mu_{\parallel} & 0 \\
					0 & 0 &  \mu_{\perp} \end{array} \right]
\]

\begin{equation}
	\begin{gathered}
		\mu_{\parallel}=f\cdot{\mu_1}+(1-f)\cdot \mu_2 \\
		\mu_{\perp}=\left(f\cdot{\mu_1^{-1}}+(1-f)\cdot \mu_2^{-1}\right)^{-1}.
	\end{gathered}
\label{eq:effmedium-mu}
\end{equation}

W powyższych wzorach przez $\varepsilon_{\parallel}$ i~$\varepsilon_{\perp}$ oznaczono odpowiednio składowe tensora przenikalność elektrycznej równoległe i~prostopadłe do kierunku prostopadłego do granicy między ośrodkami. Współczynnik $f$ nazywany współczynnikiem wypełnienia definiujemy jako $f=\frac{d_1}{d_1+d_2}$ oznacza stosunek grubości materiału o~współczynniku załamania $\sqrt{\varepsilon_1 \mu_1}$ do grubości całej komórki elementarnej.

Oznacza to, że w~ośrodku wypełnionym naprzemiennie dwoma materiałami, fala elektromagnetyczna propaguje się tak jak w~jednoosiowym materiale dwójłomnym. Osią takiego metamateriału jest dowolna prosta prostopadła do granic warstw dwu tworzących go materiałów~\cite{sihvola1999electromagnetic}.

Efektywny tensor przenikalności elektrycznej o~postaci (\ref{eq:effmedium}) nie zależy od okresu struktury periodycznej. Co więcej, wyrażenie (\ref{eq:effmedium}) stosuje się  także do warstwowych struktur nieperiodycznych wypełnionych dwoma materiałami z~tym samym współczynnikiem wypełnienia $f$. Z~drugiej strony, dla struktur periodycznych składających się z~dwóch naprzemiennie ułożonych rodzajów warstw, znana jest postać analityczna związku dyspersyjnego~\cite{pastuszczak2011optimized}, z~której można skorzystać, gdy opis (\ref{eq:effmedium}) jest niewystarczający. Poniższe równanie odnosi się do polaryzacji TM

\begin{equation}
cos(k_{B} \,a) = cos(k_{1} d_1) cos(k_{2} d_2) \\
- \frac{1}{2} \left(\frac{k_{1} \varepsilon_2}{k_{2} \varepsilon_1} + \frac{k_{2} \varepsilon_1}{k_{1} \varepsilon_2}\right)
sin(k_{1} d_1) sin(k_{2} d_2),
\label{eq_dispersion}
\end{equation}
gdzie $d_i$ i $\varepsilon_i$ (dla $i=1,2$) oznaczają grubość i przenikalność elektryczną warstw, $a=d_1+d_2$ jest okresem struktury, $k_B$ jest liczbą falową fali Blocha, $k_i = \sqrt{k_0^2 \varepsilon_i-k_{x}^2}$ jest składową wektora falowego wzdłuż osi z w i-tym ośrodku, a $k_0=2\pi/\lambda$ jest liczbą falową w próżni. 

Należy dodać, że znane są dokładniejsze sposoby wprowadzenia własności efektywnych wielowarstwy niż (\ref{eq:effmedium}) i (\ref{eq:effmedium-mu}), określane często jako nielokalne \cite{elser2007nonlocal,chebykin2011nonlocal}. Modele te nie będą wykorzystane w niniejszej rozprawie, jako że podobnie jak (\ref{eq:effmedium}) i (\ref{eq:effmedium-mu} pozwalają jedynie na przybliżony opis propagacji w strukturze warstwowej, mniej dokładny niż quasi-analityczna metoda TMM oraz metoda różnic skończonych FDTD z odpowiednio gęstym próbkowaniem. Warto na koniec wspomnieć, że także w zakresie swojej stosowalności, model efektywny nie pozwala prawidłowo opisać niektórych zjawisk. Przykładem takiego zjawiska jest tzw. potrójne załamanie światła~(ang.~trirefringence)~\cite{netti2001optical,diaz2016some}.
