\begin{figure}[tbh]
	\includegraphics[width=.5\textwidth]{images/multilayer/multilayer-3d.png}
	\caption{Schemat wielowarstwy metaliczno dielektrycznej}
	\label{fig:mulschem}
\end{figure}


Zgodnie z przedstawionymi własnościami materiałowymi, obrazowanie z~rozdzielczością przekraczającą klasyczne ograniczenie dyfrakcyjne przy pomocy metaliwiąże się z~dużymi stratami natężenia światła w wyniku absorpcji. Zwiększenie współczynnika transmisji przez wielowarstwy zawierające metal możliwe jest dzięki wykorzystaniu efektu rezonansowego tunelowania \cite{scalora-transparentmetal}. Chociaż zastosowanie zaproponowane w cytowanej pracy nie było związane z obrazowaniem, to możliwość uzyskania współczynnika transmisji rzędu 70\% dla wielowarstwy zawierającej łącznie $40$~nm srebra otwiera możliwości wysokiej transmisji i wykorzystania materiałów o ujemnym $\varepsilon$. Schemat wielowarstwy przedstawia rysunek \ref{fig:mulschem}. W proponowanym podejściu obrazowanie nadrozdzielcze nie wynika wprost z zastosowania materiału o $\varepsilon = -1$, ale z efektywnych anizotropowych właściwości powstałego w ten sposób  metamateriału \cite{ramakrishna2003imaging}. Przy pomocy przybliżenia ośrodka efektywnego, szerzej omówionego w rozdziale \ref{subart:effmedium}, możemy dobierając grubości warstw do parametrów stosowanych materiałów uzyskać metamateriał o $\varepsilon_z \to \infty$ i $\varepsilon_x \to 0$.


\begin{figure}[tbh]
	\includegraphics[width=\textwidth]{images/multilayer/agtio2-effective.png}
	\caption{Przenikalność ośrodka efektywnego obliczona zgodnie z \ref{eq:effmedium}  zbudowanego z warstw $Ag$ \cite{PhysRevB.6.4370} i~$TiO_2$ \cite{DEVORE:51}. Współczynnik wypełnienia f=1 oznacza, że struktura zbudowana jest jedynie ze srebra. Przy pomocy konturu zaznaczono $\varepsilon_x=0$ oraz $\varepsilon_z=100$.}
	\label{fig:multiex}
%generacja rysuknu:
%./effEpsilon.py database/main/Ag/Johnson.yml database/main/TiO2/Devore-o.yml
\end{figure}

Przykład materiałów z których w opisany sposób można konstruować wielowarstwę charateryzującą się transmisją bezdyrakcyjną prezentują wykresy na rysunku \ref{fig:multiex}.~W szczególności na wykresach zaobserwować możemy, że obszar wysokiego $\varepsilon_z$ graniczy z obszarem w którym ta składowa przenikalności elektrycznej przyjmuje wartości ujemne. Dla uzyskania własności bezdyfrakcyjnych, kluczowe jest dobranie takiego współczynnika wypełnienia~$f$, który pozwoli dla określonych wybranych długości fali uzyskać efektywne wartości składowych tensora przenikalności elektrycznej jak najbliższe tym przez nas poszukiwanym.~W przypadku prezentowanych materiałów dla długości fali ok.~$500$~nm możemy uzyskać $\varepsilon_x \approx 0$ i~$\varepsilon_z \approx 10$
\footnote{TODO: ew. FDTD dla przeliczonego $Ag$ i $TiO_2$}







