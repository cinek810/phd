Wykorzystywane w~optyce materiały charakteryzują się niską podatnością magnetyczną w~rozważanej części widma, w~związku z~czym przyjmuje się ${\mu(\omega)=1}$. Ze względu na właściwości elektryczne  materiały te możemy  podzielić na dielektryki~i przewodniki. Dielektrykami nazywamy materiały,~w których pod wpływem zewnętrznego pola elektrycznego powstają dipole elektryczne. Powodem powstawania dipoli może być przesunięcie ładunków dodatnich w~stosunku do ujemnych lub powstanie spójnej orientacji przestrzennej dipoli elektrycznych tworzących dany ośrodek. W przeciwieństwie do dielektryków, ze względu na obecność swobodnych nośników ładunku elektrycznego przewodniki nie ulegają polaryzacji w~zewnętrznym polu elektrycznym. W niniejszej pracy jako przewodniki rozważane są metaliczne pierwiastki chemiczne, dlatego terminy przewodnik i~metal traktowane są zamiennie.

\begin{figure}[tb]
	\includegraphics[width=\textwidth]{images/agn.png}
	\caption{Zależność współczynnika załamania od długości fali w~zakresie optycznym, dla metalu - $Ag$ \cite{PhysRevB.6.4370}.  }
	\label{fig:agn}
\end{figure}
Zjawiska fizyczne omawiane w~poniższym rozdziale bardzo silnie zależą od przenikalności elektrycznej wykorzystywanych materiałów. W szczególności wymagają wykorzystywania materiałów o~ujemnej przenikalności elektrycznej. Takie własności przejawiają metale, których zastosowanie do nadrozdzielczego obrazowania za pomocą cienkiej warstwy zaproponował John Pendry. Wykorzystanie warstwy znacznie cieńszej od długości fali pozwala na rozprzężenie pola elektrycznego i~magnetycznego przez co możliwe jest nadrozdzielcze obrazowanie za pomocą materiału z~$\mu=$~1\cite{PhysRevLett.85.3966}.

\begin{figure}[tb]
	\includegraphics[width=\textwidth]{images/sio2n.png}
	\caption{Zależność współczynnika załamania od długości fali w~zakresie optycznym, dla szkła kwarcowego -  $SiO_2$ \cite{MALITSON:65}   }
	\label{fig:sio2n}
\end{figure}
W zakresie optycznym znajdują się częstości rezonansowe atomów metali, co skutkuje silną dyspersją współczynnika załamania i~wysoką absorpcją w~tym zakresie. Zależność rzeczywistej i~urojonej części współczynnika załamania  dla srebra prezentuje wykres \ref{fig:agn}. Na wykresie widać charakterystyczny obszar w~zakresie ok. 310-350~nm, w~którym obserwujemy znaczny spadek części rzeczywistej współczynnika załamania i~minimum zdolności absorpcyjnych. Wysoka wartość części urojonej współczynnika załamania wskazuje na silną absorpcję promieniowania dla długości fali powyżej 350~nm.

Dla dielektryków współczynnik załamania zazwyczaj maleje wraz ze wzrostem długości fali. Zależność ta jest znacznie słabsza niż w~przypadku metali. Jako przykład na wykresie \ref{fig:sio2n} przedstawiono współczynnik załamania $SiO_2$. Urojona część współczynnika załamania dla dielektryków jest mniejsza niż w~przypadku metali. W~szczególność dla przedstawionego szkła kwarcowego w~większości zastosowań jest zaniedbywana.

\begin{figure}[tb]
	\includegraphics[width=\textwidth]{images/agtio2eps.png}
	\label{fig:agtio2eps}
	\caption{Zależność przenikalności elektrycznej od długości fali w~zakresie optycznym, dla metalu $Ag$\cite{PhysRevB.6.4370}. }  
\end{figure}
W celu opisu dyspersyjnych dielektryków z~powodzeniem stosuje się model Lorenza-Lorenza, a~w~niektórych przypadkach wartość $\varepsilon$ bywa traktowana jako stała. W przypadku metali, ze względu na wspomniany charakter rezonansowy $\varepsilon(\omega)$  musi być opisywana przy użyciu modelu Lorenza-Drudego. Dokładniejsze omówienie tego modelu znajduje się w~rozdziale \ref{subart:lorenz-drude}.

Należy zaznaczyć, że pominięty został wpływ wektora falowego na wartości $\varepsilon$ i~$\mu$. W ogólności $\varepsilon(\omega,\vec{k})$ jest funkcją zarówno częstotliwości jak i~wektora falowego, co należy rozumieć jako zależność indukcji elektrycznej $\vec{D}(t,\vec{r})$ nie tylko od historii wzbudzeń poprzedzającej interesujący nas czas $t$, ale również od wzbudzenia fali elektromagnetycznej w~otoczeniu $\vec{r'}$. Ze względu na zależność pola $\vec{D}$ od pola $\vec{E}$ w~otoczeniu, ta klasa zjawisk nazywana jest nielokalnymi. Nie można pomijać wpływu otoczenia na stan polaryzacji $\vec{P}$, gdy zmienność pola elektromagnetycznego jest znacząca na odległościach porównywalnych z~drogą swobodną elektronów w~ośrodku.



