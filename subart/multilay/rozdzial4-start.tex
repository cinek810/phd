Każdy element układu optycznego możemy wyrazić jako układ filtrujący częstotliwość i częstości przestrzenne oświetlającego ten układ źródła. Poniższy rozdział poświęcony jest modelowaniu działania wielowarstw metaliczno-dielektrynych wykorzystywanych do  budowy elementów optycznych o zaprojektowanych właśnościach filtrowania częstości przestrzennych. W przeciwieństwie do przestrzeni swobodnej, będącej filtrem dolnoprzepustowym, mogą one charakteryzować się również transmisją wysokich częstości przestrzennych, które w przestrzeni swobodnej mają charakter fal ewanescentnych. Wykorzystanie układów tego typu umożliwia konstrukcję elementów optycznych działających poza klasycznym ograniczeniem dyfrakcyjnym.

Złamanie ograniczenia dyfrakcyjnego możliwe jest dzięki zastosowaniu materiałów charakteryzujących się ujemnym załamaniem światła, rozumianym jako załamanie pod kątem skierowanym przeciwnie niż wynikałoby to z Prawa Snelliusa. Materiały takie w odniesieniu do przywołanej klasycznej formuły optyki geometrycznej muszą charakteryzować się ujemnym współczynnikiem załamania światła. Korzystając z elektrodynamiki klasycznej opisywanej równaniami Maxwella wiemy, że współczynnik załamania związany jest z przenikalnością elektryczną i magentyczną ośrodka: $n = \pm \sqrt{ \varepsilon \mu}$. Wybór dodatniej gałęzi pierwiastka jest więc konwencjonalny i musi być dostosowany do sytuacji fizycznej. Ujemna wartość współczynnika załamania światła jest równoważna ze zmianą kierunku prędkości fazowej, której zwrot jest zgodny ze zwrotem wektora falowego. Pierwszą propozycja definicji ośrodków o ujemnym współczyniku załamania była ujemna wartośc iloczynu skalarnego wektora Poyntinga i wektora falowego $\vec{P} \cdot \vec{k} < 0$ podana przez Victora Vesselago \cite{veselago1968electrodynamics}. Ze względu na tę własność materiały takie nazywane są lewoskrętnymi (ang. left-handed) gdyż w stosunku do do iloczynu $\vec{E} \times \vec{H}$ nie ma zastosowania reguła prawej dłoni, a przeciwna - lewej.

Materiały lewoskrętne muszą charakteryzować się ujemnymi wartościami $\varepsilon$ i $\mu$ dla tego samego zakresu częstotliwości. Materiały takie nie były do tej pory obserwowane w przyrodzie, eksperymentalnie dowiedziono jednak możliwość sztucznego wytworzenia metamateriałów o takich własnościach\cite{PhysRevLett.84.4184} przy pomocy SSR(ang split-ring resonator). 

Język używany do opisu działania analizowanych struktur warstwowych wywodzi się z Optyki Fourierowskiej w której podstawowym pojęciem są układy LSI (ang. Linear shift-invariant systems). Opisywane struktury spełniają warunki tego typu układów - nie wykazują własności nieliniowych, oraz są niezmiennicze ze względu na przesunięcia. Wykorzystanie formalizmu Optyki Fourierowskiej umożliwia analityczną ocenę wyników symulacji numerycznych, oraz wprawdza spójny zestaw pojęć wykorzystywanych do opisu rozważanych układów.

