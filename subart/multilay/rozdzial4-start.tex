Każdy element liniowego układu optycznego możemy wyrazić jako układ filtrujący częstotliwość i~częstości przestrzenne oświetlającego ten układ źródła. Poniższy rozdział poświęcony jest modelowaniu propagacji światła przez wielowarstwy metaliczno-dielektryczne wykorzystywane do  budowy elementów optycznych o~zaprojektowanych własnościach filtrowania częstości przestrzennych. W przeciwieństwie do przestrzeni swobodnej, będącej przestrzennym filtrem dolnoprzepustowym, mogą one charakteryzować się również transmisją wysokich częstości przestrzennych, które w~przestrzeni swobodnej mają charakter fal ewanescentnych. Wykorzystanie układów tego typu umożliwia konstrukcję elementów optycznych działających poza klasycznym ograniczeniem dyfrakcyjnym.

Złamanie ograniczenia dyfrakcyjnego możliwe jest dzięki zastosowaniu materiałów charakteryzujących się ujemnym załamaniem światła, rozumianym jako załamanie pod kątem skierowanym przeciwnie niż wynikałoby to z~prawa Snelliusa. Korzystając z~elektrodynamiki klasycznej opisywanej równaniami \nohyphens{Maxwella} wiemy, że współczynnik załamania związany jest z~przenikalnością elektryczną i~magnetyczną ośrodka $n = \pm \sqrt{ \varepsilon \mu}$. Wybór dodatniej gałęzi pierwiastka jest więc konwencjonalny i~musi być dostosowany do sytuacji fizycznej. Ujemna wartość współczynnika załamania światła jest równoważna zmianie kierunku prędkości fazowej, której zwrot jest zgodny ze zwrotem wektora falowego. Pierwszą propozycją definicji ośrodków o~ujemnym współczynniku załamania stanowiły ośrodki z ujemną wartością iloczynu skalarnego wektora Poyntinga i~wektora falowego $\vec{P} \cdot \vec{k} < 0$ wprowadzone przez Wiktora Wiesiełago \cite{veselago1968electrodynamics}. Ze względu na tę własność materiały takie nazywane są lewoskrętnymi (ang. left-handed) gdyż w~stosunku do iloczynu $\vec{E} \times \vec{H}$ nie ma zastosowania reguła prawej dłoni, a przeciwna - lewej.

Materiały lewoskrętne muszą charakteryzować się ujemnymi wartościami $\varepsilon$ i~µ dla tego samego zakresu częstotliwości. Materiały takie nie były do tej pory obserwowane w~przyrodzie. Eksperymentalnie dowiedziono jednak możliwości sztucznego wytworzenia metamateriałów o~takich właściwościach\cite{PhysRevLett.84.4184} za pomocą rezonatorów SRR~(ang split-ring resonator). W kolejnych latach pojawiało się wiele propozycji uzyskiwania metamateriałów, takich jak sieci z~otworami (ang. \textit{fishnet}) czy też struktury warstwowe.

Język używany do opisu działania analizowanych struktur warstwowych wywodzi się z~optyki fourierowskiej, w~której jednym z podstawowych pojęć są układy LSI (ang. Linear shift-invariant systems). Opisywane struktury spełniają warunki tego typu układów - nie wykazują własności nieliniowych, oraz są niezmiennicze ze względu na przesunięcia. Wykorzystanie formalizmu optyki fourierowskiej umożliwia analityczną ocenę wyników symulacji numerycznych, oraz wprowadza spójny zestaw pojęć wykorzystywanych do opisu rozważanych układów. Dokładniejsze omówienie podstawowych pojęć związanych z~układami LSI znajduje się w~rozdziale \ref{art:lsi}.


