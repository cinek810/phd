Dotychczas zakładaliśmy, że granice między ośrodkami tworzącymi wielowarstwę są idealnie płaskie. W warunkach eksperymentalnych, przy wykorzystaniu technik umożliwiających naprzemienne układanie kilkunastu warstw różnych materiałów o~grubości od kilkunastu do kilkudziesięciu nanometrów,  takich jak fizyczne osadzanie z~fazy gazowej (ang.~PVD - physical vapour deposition), uzyskanie doskonale płaskich warstw jest niemożliwe. W poniższym rozdziale przeanalizowany zostanie wpływ niedoskonałości warstw na obrazowanie przez struktury MDM.

Podstawowym parametrem wykorzystywanym do opisu chropowatości jest średnia kwadratowa różnic faktycznej grubości warstwy od średniej (ang.~RMS - root mean square), wyrażana wzorem:
\begin{equation}
\textrm{RMS}=\sqrt{\sum_i^n \frac{(x_i -x_0)^2}{n}},
\end{equation}
gdzie $x_i$ są kolejnymi zmierzonymi grubościami, $x_0$ grubością średnią, a $n$ odpowiada liczbie punktów, w~których wykonano pomiar. Różnice uzyskanej w~stosunku do projektowanej grubości warstwy w~blisko położonych punktach nie są zmiennymi losowymi niezależnymi, dlatego do pełnego opisu topologii powierzchni niezbędne jest wykorzystanie funkcji autokorelacji~\cite{stefaniuk2011effect}. Na podstawie pomiarów mikroskopem sił atomowych (ang.~AFM - atomic force microscope) można stwierdzić, że RMS powierzchni podlega statystyce gaussowskiej. Histogram wyników uzyskanych za pomocą pomiarów AFM przedstawia wykres \ref{fig:ag30nm-afmhist}. Dwuwymiarowy skan uzyskany w~pomiarach przedstawia rysunek \ref{fig:ag30nm-afmmeasure}.

\begin{figure}[bt]
		\includegraphics[width=\textwidth]{images/multilayer/ag30nm-afm-measure-hist.png}
		\caption{Histogram odchyleń od średniej grubości dla warstwy $30$~nm napylonej przy pomocy PVD zmierzonych za pomocą AFM} 		\label{fig:ag30nm-afmhist}
\end{figure}

\begin{figure}[bt]
		\includegraphics[width=\textwidth]{images/multilayer/ag30nm-afm-measure.png}
		\caption{Pomiary grubości na powierzchni napylonej warstwy $30$~nm srebra za pomocą AFM. Pomiary wykonał dr~Tomasz~Stefaniuk.} 
		\label{fig:ag30nm-afmmeasure}
\end{figure}


Efektywne współczynniki przenikalności elektrycznej uzyskiwane za pomocą wzoru (\ref{eq:effmedium}) w~znacznym stopniu zmieniają się w~wyniku wprowadzenia chropowatości~\cite{ludwig2012impact}. Szczególnie dużą zmienność można zaobserwować w~okolicach rezonansu dla $\varepsilon_{\perp}$, czyli~w zakresie długości fali, dla którego projektowane są własności metamateriału. Zbliżenie wartości do przewidywanych w~warunkach homogenizacji można zaobserwować w~przypadku struktur, dla których punkty odpowiadające pomiarom grubości z~mikroskopu są bardziej oddalone - próbkowanie pomiaru mikroskopowego jest rzadsze niż w~symulacji numerycznej. Ze względu na przybliżenie granicy warstwy pomiędzy punktami pomiarowymi z~AFM poprzez funkcję gładką, Ludwig i~inni otrzymują większe gładkie obszary na powierzchni symulowanej granicy między ośrodkami~\cite{ludwig2012impact}. 

\begin{figure}[!hbt]
	\begin{center}
	\includegraphics[width=.9\textwidth]{images/multilayer/plp-chropo.png}
	\end{center}
	\caption{Rozkład natężenia pola elektromagnetycznego wewnątrz i~poza strukturą warstwową o~własnościach supersoczewki z~warstwami gładkimi- (a) i (c) oraz chropowatymi- (b) i (d), oświetlonej za pomocą źródła monochromatycznego o~długościach fali odpowiednio (a),(b)~$\lambda=430$~nm  i~(c),(d) $\lambda=490$~nm~\cite{Stolarek_2013}}
	\label{fig:plp-chropo-fdtd}
\end{figure}

Nierówność warstw może mieć pozytywny wpływ na niektóre parametry opisujące zdolności obrazujące wielowarstwy. Uwzględnienie chropowatości może zwiększyć współczynnik transmisji przez granicę dwóch ośrodków poprzez skrócenie zasięgu propagacji plazmonów powierzchniowych w~przypadku przypadkowej chropowatości, oraz dodatkowe wzmocnienie fal ewanescentnych za pomocą sinusoidalnej chropowatości o~okresie podfalowym~\cite{huang2012subwavelength}. Przykład układu dla którego wprowadzenie chropowatości zwiększa współczynnik transmisji dla wąskiego zakresu długości fal, przedstawia rozkład pola elektromagnetycznego na rysunku \ref{fig:plp-chropo-fdtd}~a~i~b. W ogólności jednak wzrost chropowatości powierzchni zmniejsza współczynnik transmisji przez strukturę warstwową, co możemy zaobserwować po zmianie długości fali oświetlającej soczewkę na rozkładach pola na rysunkach \ref{fig:plp-chropo-fdtd}~c~i~d. 

\begin{figure}[bt]
		\includegraphics[width=\textwidth]{images/multilayer/ag30nm-afm-generated.png}
		\caption{Wizualizacja powierzchni chropowatej wygenerowanej na podstawie pomiarów AFM. Generowana dwuwymiarowa macierz losowa podlegające rozkładowi normalnemu o~widmowej gęstości mocy odpowiadającej wynikom pomiarów za pomocą AFM.} 
		\label{fig:ag30nm-afmgene}
%	Generator:
%	pomocnicze/wielowarstw/chropo/gen-chropo.py
\end{figure}


Zmiana właściwości materiałów, z~których zbudowana jest wielowarstwa, na charakteryzujące się mniejszą absorpcją, nie może być wykorzystana do kompensacji strat transmisji w~wyniku nierówności warstw. Wprowadzenie chropowatości prowadzi do powstania losowych zaburzeń rozkładu pola elektromagnetycznego, których interferencja wprowadza zniekształcenie optycznej funkcji przenoszenia~(ang.~OTF - Optical Transfer Function)~\cite{citeulike:2926459}. Odpowiednio dobrany współczynnik absorpcji wewnątrz metali zapewnia szybkie zanikanie losowych zaburzeń umożliwiając zachowanie płaskiego charakteru OTF. Szczególne znaczenie dla zachowania własności obrazowania podfalowego ma płaszczyzna wyjściowa wielowarstwy, na której utrzymanie RMS poniżej $0.6$~nm jest kluczowe dla uzyskania PSF o~szerokości podfalowej~\cite{guo2014negative}.

Należy zwrócić uwagę, że na skutek chropowatości współczynnik $\varepsilon_z$ zostaje zmniejszony w~okolicach rezonansu~\cite{guo2014negative} (dla idealnej supersoczewki $\varepsilon_{z} \to - \infty$), co powoduje, że możliwa jest efektywna transmisja wyższych częstości przestrzennych, a co za tym idzie zwiększenie zdolności rozdzielczej układu. Własności obrazujące, które są optymalne przy płaskim kształcie OTF zostają jednocześnie zaburzone, a ich zachowanie możliwe jest poprzez użycie materiałów o~większym współczynniku absorpcji. Na podstawie takiego rozważania Zhen Guo i~in.~\cite{guo2014negative} wnioskują, że chropowatość w~zasadniczy sposób pogarsza zdolności obrazujące supersoczewki. Zdolność rozdzielcza jest natomiast determinowana poprzez stratność użytych materiałów.

\begin{figure}
	\centering
	\begin{subfigure}[b]{.45\textwidth}
		\includegraphics[width=\textwidth]{images/multilayer/oer-rms01.png}
		\caption{$RMS=0.1$~nm}
	\end{subfigure}
	\begin{subfigure}[b]{.45\textwidth}
		\includegraphics[width=\textwidth]{images/multilayer/oer-rms05.png}
		\caption{$RMS=0.5$~nm}
	\end{subfigure}
	\caption{Wyniki symulacji wielowarstwy o~17 chropowatych granicach ośrodków, z~różnymi wartościami RMS charakteryzującymi chropowatość warstw. Na ilustracji (b) obserwujemy znaczne ograniczenie strumienia fali E-M propagującego się w~pole dalekie na skutek interferencji wielu fal płaskich losowo zaburzonych przez nierówności~\cite{pastuszczak2013engineering}.}
	\label{fig:rmsem}
	
\end{figure}

Porównanie wyników prac numerycznych prowadzonych przez różnych autorów dotyczących wpływu chropowatości na współczynnik transmisji, szerokość i~kształt PSF oraz na zdolność rozdzielczą wielowarstwy wymaga uwzględnienia różnic w~zastosowanej przez nich metodyce. Kluczowym elementem jest sposób generacji powierzchni chropowatej - w~niektórych pracach nie jest uwzględniana autokorelacja nierówności~\cite{guo2014negative} przez co zaniedbane zostają charakterystyczne elementy topologii widoczne w~pomiarach za pomocą AFM. W innych wykorzystywane są algorytmy heurystyczne łączące punkty z~pomiarów mikroskopowych za pomocą wielomianów sklejanych\footnote{tzw. krzywa B-sklejana, w~literaturze polskiej postulowana bywa również nazwa splajn od angielskigo B-spline}~\cite{ludwig2012impact}, w~innych pracach autorzy opierają się na widmowym rozkładzie gęstości mocy zmiennej losowej~\cite{pastuszczak2013engineering}. Przykład powierzchni chropowatej wygenerowanej na podstawie pomiarów z~mikroskopu AFM z~wykorzystaniem ostatniej z~wymienionych metod znajduje się na ilustracji \ref{fig:ag30nm-afmgene}.

Niezależnie od zastosowanej metodyki symulacji pola elektromagnetycznego i~generacji warstw chropowatych składających się na supersoczewki zbudowane ze struktur MDM, wyniki pozwalają na wysunięcie zgodnych wniosków. Uzyskanie nadrozdzielczego obrazowania przez omawiane układy  możliwe jest jest jedynie w~wielowarstwach o~RMS$<1.5$~nm~\citep{guo2014negative,stefaniuk2011effect,ludwig2012impact}. Ponieważ każda chropowata powierzchnia przyczynia się do rozproszenia fali, wraz ze wzrostem liczby warstw własności transmisyjne i~obrazujące stosu MDM stają się bardziej wrażliwe na chropowatości powierzchni \cite{guo2014negative}. W przypadku stosów składających się z~kilkunastu warstw, RMS nawet na poziomie $0.5$~nm może uniemożliwić uzyskanie wysokiego współczynnika transmisji, a co za tym idzie praktycznego wykorzystania tego typu soczewek~\cite{pastuszczak2013engineering}. Wpływ chropowatości na rozkład pola E-M jak i~współczynnik transmisji przez stos metaliczno-dielektryczny prezentują rozkłady pola E-M na rysunku \ref{fig:rmsem}.




%%%%%%%%Obrazki z~publikacji w~PLP - wyniki pomoarow afm w~1d i~2d
%%%%%%%%\begin{figure}
%%%%%%%%		\includegraphics[width=\textwidth]{images/multilayer/plp-afm-chropo-1d.png}\\
%%%%%%%%\end{figure}

%%%%%%%%\begin{figure}
%%%%%%%%		\includegraphics[width=\textwidth]{images/multilayer/plp-afm-chropo.png}\\
%%%%%%%%\end{figure}



