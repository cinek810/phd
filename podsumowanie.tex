\chapter{Podsumowanie}
Wspólnym mianownikiem pracy są wykorzystywane metody, oraz podfalowa charakterystyka analizowanych elementów. Warto jednak zwrócić uwagę, że w w poszczególnych rozdziałach analizie poddawane były zupełnie inne zakresy długości fali, charakteryzujące się innymi własnościami materii. Zaczynając od układów do jednokierunkowej transmisji i~skupiania wiązki światła dla zakresu THz~(długości fali ok.~3~cm), które zostały przedstawione w~rozdziale~\ref{chap:thz}. 

Prace te koncentrowały się na projektowaniu i~optymalizacji układów, które podlegały późniejszej weryfikacji eksperymentalnej. Wyniki tych prac wskazały na możliwość uzyskania jednokierunkowej transmisji, zgodnej z~twierdzeniem o~wzajemności, w~-1~i~+1 rzędzie dyfrakcyjnym. Przeprowadzona analiza teoretyczna, wraz z potwierdzeniem eksperymentalnym, stanowiły podstawę krytki publikowanych wcześniej prac donoszących o możliwej asymetrii w zerowym rzędzie ugięcia~\cite{Stolarek:13}. W kolejnym kroku posługując się wiedzą zdobytą na temat działania siatek DMG, zaproponowana została jednokierunkowa soczewka dyfrakcyjna charakteryzująca się kontrastem $C=99.8\%$. 

Dla zakresu THz zaprojektowane zostały również koncentryczne siatki dyfrakcyjne, w których podkład z $GaAs$ mozna traktować, jako rdzeń falowodu. Konstrukcja ta, pozwoliła na budowę efektywnych anten dla detektorów promieniowania THz, omówionych w~\ref{subart:antenaThz}. Zaprojektowane anteny pozwalają na sprzęganie fal THz z obszarów o rozmiarach kilku centymetrów kwadratowych, z wydajnością  80\%.

W~rozdziale \ref{roz:pml} prowadzone prace realizowane były dla zakresu dalekiej podczerwieni. Przedstawiono prace numeryczne zawierające propozycję realizacji nie odbijającej warstwy pochłaniającej za pomocą układów warstwowych. Całość projektu przedstawia analizę opartą na wyidealizowanych~(nieistniejących fizycznie) materiałach, przez serię przybliżeń, aż do symulacji opartych na własnościach materiałowych zaczerpniętych z~prac eksperymentalnych~\cite{ania2015,stefaniuk2015perfectly}.

W przedostatnim,  rozdziale \ref{art:nondiff}  omówione zostały struktury fotoniczne dla światła widzialnego~(długości fali rzędu kilkuset~nanometrów). Przedstawiono, w~szerokim kontekście literaturowym, wkład autora w~badania dotyczące układów opartych o~wielowarstwy metaliczno-dielektryczne przeznaczone do obrazowania, rzutowania i~koncentracji wiązek promieniowania E-M o~rozmiarach podfalowych. Wykonane symulacje o bardzo dużej rozdzielczości pozwoliły określić wymagania na parametry statystyczne opisujące gładkość napylonych warstw umożliwiającą eksperymentalną weryfikację obrazowania podfalowego.

Ze względu na duże wymogi mocy obliczeniowej jak i złożoność pamięciową wykorzystywanych metod konieczne było wykorzystanie infrastruktury HPC~(ang. High Performance Computing). Używane zasoby obliczeniowe były udostępniane przez Interdyscyplinarne Centrum Modelowania Matematycznego i~Komputerowego UW oraz infrastrukturę PLgrid. 
