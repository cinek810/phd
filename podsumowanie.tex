\chapter{Podsumowanie}
Niniejsza praca jest wynikiem kilku lat studiów doktoranckich autora, w~trakcie których prowadził on symulacje, których celem było projektowanie i~optymalizacja struktur o~rozmiarach podfalowych. Podstawowym narzędziem wykorzystywaną przez autora były obliczenia numeryczne metodą FDTD za pomocą aplikacji meep\cite{OskooiRo10} na komputerach dużej mocy udostępnianych w~ramach Interdyscyplinarnego Centrum Modelowania Matematycznego i~Komputerowego UW oraz infrastruktury PLgrid. Pomimo zastosowania podobnych metod obliczeniowych, przedstawione w~różnych rozdziałach układy badane były dla różnych zakresów długości fali.

Zaczynając od układów do jednokierunkowej transmisji i~skupiania wiązki światła dla zakresu THz (długości fali ok.~3~cm), które zostały przedstawione w~rozdziale \ref{chap:thz}. Prace te koncentrowały się na projektowaniu i~optymalizacji układów, które podlegały późniejszej weryfikacji eksperymentalnej. Wyniki tych prac wskazały na możliwość uzyskania jednokierunkowej transmisji, zgodnej z~twierdzeniem o~wzajemności, w~-1~i~+1 rzędzie dyfrakcyjnym.

Przez zakres podczerwony, dla którego w~rozdziale \ref{roz:pml} przedstawiono prace numeryczne zawierające propozycję realizacji nie odbijającej warstwy pochłaniającej przy pomocy układów warstwowych. Całość projektu przedstawia analizę opartą na wyidealizowanych~(nieistniejących fizycznie materiałach), przez serię przybliżeń, aż do symulacji opartych na własnościach materiałowych zaczerpniętych z~prac eksperymentalnych.

W przedostatnim  rozdziale \ref{art:nondiff},  omówione zostały struktury fotoniczne dla światła widzialnego~(długości fali rzędu kilkuset~nanometrów). Przedstawiono, w~szerokim kontekście literaturowym, wkład autora w~badania dotyczące układów opartych o~wielowarstwy metaliczno-dielektryczne przeznaczone do obrazowania, rzutowania i~koncentracji wiązek promieniowania E-M o~rozmiarach podfalowych.

