\chapter{Podsumowanie}
Niniejsza rozprawa jest wynikiem kilku lat studiów doktoranckich autora. Podstawowym sposobem pracy autora były symulacje numeryczne, których celem było projektowanie i~optymalizacja struktur o~rozmiarach podfalowych. Narzędziem codziennej pracy była aplikacja meep~\cite{OskooiRo10} umożliwiająca wykonywanie obliczeń metodą FDTD. Ze względu na duże wymogi mocy obliczeniowej jak i złożoność pamięciową metody konieczne było wykorzystanie infrastruktury HPC~(ang. high performance computing). Używane zasoby obliczeniowe były udostępniane w~ramach Interdyscyplinarnego Centrum Modelowania Matematycznego i~Komputerowego UW oraz infrastruktury PLgrid. Techniczne aspekty optymalizacji wykorzystywanych aplikacji nie były tematem powyższej rozprawy doktorskiej, stanowiły jednak niezbędny element, bez którego analiza zjawisk fizycznych nie byłaby możliwa.

Wspólnym mianownikiem pracy są wykorzystywane metody, oraz podfalowa charakterystyka analizowanych elementów. Warto jednak zwrócić uwagę, że w w poszczególnych rozdziałach analizie poddawane były zupełnie inne zakresy długości fali, charakteryzujące się innymi własnościami materii. Zaczynając od układów do jednokierunkowej transmisji i~skupiania wiązki światła dla zakresu THz~(długości fali ok.~3~cm), które zostały przedstawione w~rozdziale~\ref{chap:thz}. 

Prace te koncentrowały się na projektowaniu i~optymalizacji układów, które podlegały późniejszej weryfikacji eksperymentalnej. Wyniki tych prac wskazały na możliwość uzyskania jednokierunkowej transmisji, zgodnej z~twierdzeniem o~wzajemności, w~-1~i~+1 rzędzie dyfrakcyjnym. 

Przez zakres podczerwony, dla którego w~rozdziale \ref{roz:pml} przedstawiono prace numeryczne zawierające propozycję realizacji nie odbijającej warstwy pochłaniającej za pomocą układów warstwowych. Całość projektu przedstawia analizę opartą na wyidealizowanych~(nieistniejących fizycznie materiałach), przez serię przybliżeń, aż do symulacji opartych na własnościach materiałowych zaczerpniętych z~prac eksperymentalnych.

W przedostatnim  rozdziale \ref{art:nondiff},  omówione zostały struktury fotoniczne dla światła widzialnego~(długości fali rzędu kilkuset~nanometrów). Przedstawiono, w~szerokim kontekście literaturowym, wkład autora w~badania dotyczące układów opartych o~wielowarstwy metaliczno-dielektryczne przeznaczone do obrazowania, rzutowania i~koncentracji wiązek promieniowania E-M o~rozmiarach podfalowych. Wykonane symulacje o bardzo dużej rozdzielczości pozwoliły określić wymagania na parametry statystyczne opisujące gładkość napylonych warstw umożliwiającą eksperymentalną weryfikację obrazowania podfalowego.

