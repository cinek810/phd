\chapter*{Skorowidz skrótowców}
\noindent ABC - ang. absorbing boundary condition, absorbujący warunek brzegowy
\\AFM - ang. atomic force microscope, mikroskop sił atomowych
\\BOR FDTD - ang. body of revolution FDTD, nazwa algorytmu FDTD służącego do prowadzenia symulacji dla struktur o symetrii cylindrycznej 
\\BPM - ang. beam propagation method, metoda propagacji wiązki
\\DMG - and. double metallic grating, podwójna metalowa siatka dyfrakcyjna
\\EMA - ang. effective medium approximation, przybliżenie ośrodka efektywnego
\\EMT - ang. effective medium theory, teoria ośrodka efektywnego
\\FDTD - ang. finite difference time domain, metoda różnic skończonych w dziedzinie czasu
\\FEM - ang. finite-element method, metoda elementu skończonego
\\FWHM - ang. full width at half maximum, szerokość połówkowa
\\HPC - ang. high performance computing, obliczenia z wykorzystaniem komputerów dużej mocy
\\LSI - ang linear shift-invariant, liniowy niezmienniczy ze względu na przesunięcia
\\OTF - ang. optical transfer function, optyczna funkcja przenoszenia
\\PEC - ang. perfect electric conductor, doskonały przewodnik
\\PMC - ang. perfect magnetic conductor, doskonały przewodnik magnetyczny 
\\PML - ang. perfectly matched layer, warstwa na granicy z którą nie występuje zjawisko odbicia
\\PSF - ang. point spread function, funkcja rozmycia punktu
\\PTFE - politetrafluoroetylen, teflon
\\PVD - ang. physical vapour deposition, fizycznie osadzenie z fazy gazowej
\\RAM - ang. random access memory, pamięć o dostępie swobodnym
\\RCWA - ang. rigorous coupled wave analysis, półanalityczna metoda obliczeniowa implementowana w przestrzeni fourierowskiej
\\RMS - ang. root mean square, średnia kwadratowa
\\SMG - ang. single metallic grating, pojedyncza siatka metalowa - termin stosowany dla podkreślenia różnicy w stosunku do DMG
\\SMM - ang. scattering matrix method, metoda obliczeniowa wykorzystująca tzw. macierze rozpraszania
\\SNR - ang. signal to noise ratio, stosunek sygnału do szumu
\\SP - ang. surface plasmon, plazmon powierzchniowy
\\SPP - ang. surface plasmon polariton, powierzchniowy plazmon-polaryton
\\SRR - ang. split-ring resonator, popularny układ wykorzystywany do projektowania metamateriałów dla fal elektromagnetycznych z zakresu mikrofalowego
\\TE - ang. transverse electric, określenie polaryzacji fali elektromagnetycznej, w której pole elektryczne drga równolegle do rozważanej płaszczyzny
\\TFSF - ang. total field scattered field, rodzaj źródła, w którym nieodbijające warunki brzegowe są uzyskiwane poprzez podział całości obszaru symulacji na pola całkowite i rozproszone
\\TM - ang. transverse magnetic, określenie polaryzacji fali elektromagnetycznej, w której pole magnetyczne drga równolegle do rozważanej płaszczyzny
\\TMM - ang. transfer matrix method, metoda obliczeniowa wykorzystująca tzw. macierze przejścia
\\UPML - ang.  uniaxial perfectly matched layer, materiał typu PML realizowany przy pomocy ośrodka anizotropowego
\\UW - Uniwersytet Warszawski
